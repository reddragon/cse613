\documentclass{article}
\usepackage{graphicx}
\usepackage{amssymb}
\usepackage{amsmath}
\usepackage{tikz}
\usepackage{url}
\usepackage{color}
\usepackage{savetrees}
% \usetikzlibrary{shapes}


\linespread{1.5}
\setlength{\parindent}{0pt}
\setlength{\parskip}{1.9ex plus 0.5ex minus 0.2ex}

% \linespread{1.4}

\title{CSE613\\Parallel Programming -- Homework-1}

\author{Dhruv Matani (108267786) \& Gaurav Menghani (108266803)}

\renewcommand{\thesubsection}{\thesection\ (\alph{subsection})}

\begin{document}
\maketitle

\clearpage

\tableofcontents

\clearpage

\section{Pairwise Sequence Alignment with Affine Gap Costs}

Compiling all the programs:
\begin{verbatim}
$ cd Prob-1
$ make
\end{verbatim}

\subsection{Serial algorithm}

Please check archive for the code

Running the code:
\begin{verbatim}
$ a/a.out < input_file_path
\end{verbatim}

\subsection{Parallelized algorithm by parallelizing the computation of the diagonal}

Running the code:
\begin{verbatim}
$ b/a.out [string size in bytes in the input file (this is a power of 2)] 1 < input_file_path
\end{verbatim}

\textit{Analysis of the resulting parallelism:} $T_{\infty} = \log{1} + \log{2} + \log{3} + \log{4} + \ldots{} + \log{n}$

$\Rightarrow T_{\infty} = \log{n!}$

$\Rightarrow T_{\infty} = n\log{n}$

$\therefore$ Parallelism = $\dfrac{n^2}{n\log{n}} = \dfrac{n}{\log{n}}$

\subsection{Parallel Divide \& Conquer algorithm}

Running the code:
\begin{verbatim}
$ c/a.out [base case size (this is a power of 2 - default: 16)] < input_file_path
\end{verbatim}

The best running time we got was for a base case of a block of size $32 \times 32$ elements.

\subsection{Parallelism and space usage of the Parallel Divide \& Conquer algorithm}

$T_{\infty}(n) = 3T_{\infty}(\frac{n}{2}) + O(1)$

$\Rightarrow T_{\infty}(n) = n^{\log_2{3}} = n^{1.58}$

$\therefore$ Parallelism = $\dfrac{n^2}{n^{1.58}} = n^{0.42}$

Space usage is $O(n^2)$.

\subsection{Boundary Generation parallel LCS algorithm}

The best running time we got was for a base case of a block of size $FIXME$ elements.

Running the code:
\begin{verbatim}
$ e/a.out [base case size (this is a power of 2 - default: 16)] < input_file_path
\end{verbatim}

\subsection{Parallelism and space usage of the Boundary Generation parallel LCS algorithm}

$T_{\infty}(n) = \log{1}Q(\frac{n}{q}) + \log{2}Q(\frac{n}{q}) + \log{3}Q(\frac{n}{q}) + \log{4}Q(\frac{n}{q}) + \ldots{} + \log{\frac{n}{q}}Q(\frac{n}{q})$

But, $Q(m) = m^{\log_2{3}}$, which is the time to solve a problem of size $m$ using the Divide \& Conquer algorithm.

$\therefore T_{\infty}(n) = Q(\dfrac{n}{q})q\log{(q+1)}$

$\Rightarrow T_{\infty}(n) = q\log{(q+1)}\left(\dfrac{n}{q}\right)^{\log_2{3}}$

$\therefore$ Parallelism = $\dfrac{n^2}{q\log{(q+1)}(\frac{n}{q})^{\log_2{3}}} = \dfrac{n^2}{q\log{(q+1)}(\frac{n}{q})^{\log_2{3}}}$

Space usage is $O(n^2)$.

\subsection{Limit on $n$ and Cilkview scalability plots}

The limit on $n$ we experimentally determined happens to be ${n'}_{max} = 32768$.

\begin{figure}
  \begin{center}
    \includegraphics[width=4in]{images/agc-partc.png}
    \caption{Cilkview plot for part(c) - Divide \& Conquer solution}
    \label{fig:cilkview_c}
  \end{center}
\end {figure}

\begin{figure}
  \begin{center}
    \includegraphics[width=4in]{images/agc-parte.png}
    \caption{Cilkview plot for part(e) - Cache Efficient blocking-based diagonal solution}
    \label{fig:cilkview_e}
  \end{center}
\end {figure}


\subsection{Limit on $n$, \texttt{PAPI} for number of L1 cache misses for the parallel \& serial implementation}
Number of L1 Misses for the Serial Implementation: 224395558\\
Number of L2 Misses for the Serial Implementation: 7937023\\

Number of L1 Misses for the Serial Implementation: 224395558\\
Number of L2 Misses for the Serial Implementation: 7937023





\clearpage

\section{Pairwise Alignment for Highly Uneven Sequence Lengths}

\clearpage

\section{Randomized Parallel Quicksort with Looping}

\subsection{Upper bound on recursion depth of \texttt{PAR-RANDOMIZED-LOOPING-QUICKSORT}}

We know that the routine \texttt{PAR-PARTITION} randomly splits its
input, but the routine \texttt{PAR-RANDOMIZED-LOOPING-QUICKSORT}
doesn't continue unless the split is at least between
$\frac{1}{4}^{th}$ \& $\frac{3}{4}^{th}$.

What this means is that the upper bound on the depth of
\texttt{PAR-RANDOMIZED-LOOPING-QUICKSORT} is determined by the
following recurrence:

$T(n) = T(\frac{3n}{4}) + 1$

$\Rightarrow T(n) = \log_{4/3}{n}$

$\Rightarrow T(n) = \dfrac{\ln{n}}{\ln{4/3}}$

$\Rightarrow T(n) = 3.48\ln{n} < 8\ln{n}$

\subsection{Proof of expected depth and work}

Since \texttt{PAR-PARTITION} chooses a pivot at random, the
probability of choosing a \textit{good} pivot (i.e. a pivot that
splits the input into at most $\frac{1}{4}^{th}$ \&
$\frac{3}{4}^{th}$) is $\frac{1}{2}$.

Hence, it is \textit{expected} that only half the times will
\texttt{PAR-PARTITION} choose a bad pivot. The other half times, it
will choose a good pivot. Therefore, the expected number of times
\texttt{PAR-PARTITION} will be called before it chooses a
\textit{good} pivot is:

$E[number\ of\ calls] = \frac{1}{2} \times 1 + \frac{1}{2}^2 \times 2 +
\frac{1}{2}^3 \times 3 + \frac{1}{2}^4 \times 4 + \ldots$

$\Rightarrow E[number\ of\ calls] = 2$

Therefore, \texttt{PAR-PARTITION} splits its input into a constant
fraction at most every other time it is called (in an expected
sense). This makes the depth of
\texttt{PAR-RANDOMIZED-LOOPING-QUICKSORT} $O(\log^{3}{n})$, and the
work done $O(n\log{n})$.

\subsection{Tracking an Element}

We know that the depth of \texttt{PAR-RANDOMIZED-LOOPING-QUICKSORT}
(excluding that of \texttt{PAR-PARTITION}) is $< 4\ln{n}$.

The expected number of times we call \texttt{PAR-PARTITION} per call
of \texttt{PAR-RANDOMIZED-LOOPING-QUICKSORT} is 2.

Hence, the expected number of times an element $v$ of the array
features in an array passed to \texttt{PAR-PARTITION} is $R_v =
8\ln{n}$.

We shall now use Chernoff Bounds. $E[x] = 8\ln{n}$. We need to show
that $R_v < 32\ln{n}$ with high probability.

$\therefore Pr(X > (1 + \delta)E[X]) \le 
\left(\dfrac{e^\delta}{(1+\delta)^{(1+\delta)}}\right)^{E[X]}$

Here, $(1 + \delta)E[X] = 32\ln{n}$

$\therefore (1+\delta)8\ln{n} = 32\ln{n}$

$\Rightarrow (1+\delta) = 4$

$\Rightarrow \delta = 3$

$\therefore Pr(X > (1 + \delta)E[X]) \le 
\left(\dfrac{e^3}{4^4}\right)^{8\ln{n}}$

$\le \left(\dfrac{1}{12.75}\right)^{8\ln{n}}$

$\le \left(\dfrac{1}{698363412.02}\right)^{\ln{n}}$

$\le \left(\dfrac{1}{n}\right)^{\ln{698363412.02}}$

$\le \left(\dfrac{1}{n}\right)^{20.36}$

$\le \dfrac{1}{n^{20.36}} \le \dfrac{1}{n^{2}}$

\subsection{Proof of depth and work with high probability}

In part $(3.3)$ above, we showed that a randomly chosen element
doesn't feature is more than $32\ln{n}$ calls to
\texttt{PAR-PARTITION} with high probability.

Hence, the depth and work of \texttt{PAR-PARTITION} is $O(\ln^2{n})$
and $O(n)$ respectively with high probability.

This makes the depth and work of
\texttt{PAR-RANDOMIZED-LOOPING-QUICKSORT} $O(\ln^3{n})$ and
$O(n\ln{n})$ respectively with high probability.


\end{document}
