\documentclass{article}
\usepackage{graphicx}
\usepackage{amssymb}
\usepackage{amsmath}
\usepackage{tikz}
\usepackage{url}
\usepackage{color}
% \usepackage{savetrees}
\usepackage{listings}
% \usetikzlibrary{shapes}

% pseudocode notations
\newcommand{\xif}{{\bf{\em{if~}}}}
\newcommand{\xthen}{{\bf{\em{then~}}}}
\newcommand{\xelse}{{\bf{\em{else~}}}}
\newcommand{\xelseif}{{\bf{\em{elif~}}}}
\newcommand{\xfi}{{\bf{\em{fi~}}}}
\newcommand{\xcase}{{\bf{\em{case~}}}}
\newcommand{\xendcase}{{\bf{\em{endcase~}}}}
\newcommand{\xfor}{{\bf{\em{for~}}}}
\newcommand{\xto}{{\bf{\em{to~}}}}
\newcommand{\xby}{{\bf{\em{by~}}}}
\newcommand{\xdownto}{{\bf{\em{downto~}}}}
\newcommand{\xdo}{{\bf{\em{do~}}}}
\newcommand{\xrof}{{\bf{\em{rof~}}}}
\newcommand{\xwhile}{{\bf{\em{while~}}}}
\newcommand{\xendwhile}{{\bf{\em{endwhile~}}}}
\newcommand{\xand}{{\bf{\em{and~}}}}
\newcommand{\xor}{{\bf{\em{or~}}}}
\newcommand{\xerror}{{\bf{\em{error~}}}}
\newcommand{\xreturn}{{\bf{\em{return~}}}}
\newcommand{\xparallel}{{\bf{\em{parallel~}}}}
\newcommand{\T}{\hspace{0.5cm}}
\newcommand{\m}{\mathcal}

\def\func#1{\textrm{\bf{\sc{#1}}}}
\def\funcbf#1{\textrm{\textbf{\textsc{#1}}}}


\linespread{1.5}
\setlength{\parindent}{0pt}
\setlength{\parskip}{1.9ex plus 0.5ex minus 0.2ex}

% \linespread{1.4}

\title{CSE613\\Parallel Programming -- Homework-3}

\author{Dhruv Matani (108267786) \& Gaurav Menghani (108266803)}

\renewcommand{\thesubsection}{\thesection\ (\alph{subsection})}

\begin{document}
\maketitle

\clearpage

\tableofcontents

\clearpage

\section{Shared Memory Quicksort}

\subsection{Best base case for part(c)}

\begin{center}
  \begin{tabular}{| l | r | r | r | r | r | r | r | r | r |}
    \hline
    Base Case (\# of elements) & 4 & 16 & 32 & 64 & 128 & 512 & 1024 & 4096 & 8192 \\ \hline
    Running time (second) & 50 & 40 & 39 & 37 & 32 & 33 & 36 & 30 & 35 \\ \hline
  \end{tabular}
\end{center}

We can see that the best running time is with a base case of \texttt{4096} elements.

\subsection{Parallel Quicksort from part(c) when run on a single core and on all cores}

\begin{center}
  \begin{tabular}{| l | r | r | r | r | r | r |}
    \hline
    Test \# & 01 & 02 & 03 & 04 & 05 & 06 \\ \hline
    Serial (sec)           & 0.32 & 5.08 & 33.08 & 39.93 & 58.12 & 69.56 \\ \hline
    Parallel (sec)/speedup & 0.21/35\% & 3.08/39\% & 16.56/50\% & 17.28/57\% & 21.40/63\% & 42.39/39\% \\ \hline
  \end{tabular}
\end{center}

\subsection{Parallel Quicksort with Serial Partition}

\begin{center}
  \begin{tabular}{| l | r | r | r | r | r | r | r | r | r |}
    \hline
    Base Case (\# of elements) & 4 & 16 & 32 & 64 & 128 & 512 & 1024 & 4096 & 8192 \\ \hline
    Parallel Partition (sec) & 50 & 40 & 39 & 37 & 32 & 33 & 36 & 30 & 35 \\ \hline
    Serial Partition (sec) & 49 & 40 & 38 & 33 & 35 & 31 & 30 & 32 & 38 \\ \hline
  \end{tabular}
\end{center}


\section{Distributed Sample Sort with Shared-Memory QuickSort}

\subsection{Each process sorts $\frac{n}{p}$ + $\frac{n}{q}$ keys in worst case}
In step 1, we distribute the $n$ keys evenly (as evenly as possible)
between $p$ processes. Thus, each process has $\frac{n}{p}$ (roughly) keys.

During step 2, each process has $q-1$ local pivots, which pivot the 
$\frac{n}{p}$ keys of the process evenly. Thus each local pivot has 
about $\frac{n}{pq}$ elements to its either side. 

There are $p$ global pivots out of the total $p(q-1)$ local pivots.
Each global pivot has $q-1$ local pivots to its either side, thus, 
in step 5, when a process gets all keys between the pivots $p_i$ and 
$p_i+1$, there would be $q-1$ local pivots between those two global
pivots. Since, these local pivots were separated by 
$\frac{n}{pq}$ keys, in all during step 5, each processor would get
$q \times \frac{n}{pq}$, which is $\frac{n}{p}$ keys to sort.

TODO: Is this right?

<<<<<<< HEAD
\subsection{Implementation of the Algorithm}

\subsection{Communication \& Computation Complexities}
The communication cost can be accounted for, as follows:
\begin{enumerate}
\item Initial Distribution: $n$ keys are evenly distributed to processes. So each process
receives at max $\lceil\frac{n}{p}\rceil$ keys, and the communication cost (for sending to $p$ processes) is
$p(t_s + t_w\lceil\frac{n}{p}\rceil)$.
\item Pivot Selection: The master receives $q-1$ local pivots from each of the $p$ processes, 
and sends $p-1$ global pivots to each of the processes. This costs $p(t_s + t_w(q-1)) + 
\log{p}(t_s + t_w(p-1))$.
\item Local Bucketing: This stage has no communication cost.
\item Distribute Local Buckets: During the Initial Distribution, each process has $\frac{n}{p}$
keys. In an expected sense, after distribution of local buckets, each process should have 
the $\frac{n}{p}$ keys as well. But if the keys were randomly distributed, each process would have
contributed a roughly equal fraction of the keys to the current process' set of keys. Thus, out of
the final $\frac{n}{p}$ keys, a fraction of $\frac{n}{p^2}$ came from each of the $p-1$ processors,
and the remaining fraction of $\frac{n}{p^2}$ was with itself. Therefore, the total cost of 
communication being $(p-1)(t_s + t_w(\frac{n}{p^2}))$ in this step.
\item Local Sort: This stage has no communication cost.
\item Final Collection: In an expected sense, the sorted buckets at each process would be of size
$\frac{n}{p}$, thus the cost of getting these buckets from the processes would be
$p(t_s + t_w(\frac{n}{p}))$.
\end{enumerate}
Therefore, the total cost of communication would be: $p(t_s + t_w\lceil\frac{n}{p}\rceil) + p(t_s + t_w(q-1)) + 
\log{p}(t_s + t_w(p-1)) + (p-1)(t_s + t_w(\frac{n}{p^2})) + p(t_s + t_w(\frac{n}{p}))$.

\subsection{Value of $k$ that gives best running time is: 10}

The value of $k$ that gives us the best running time is $10$.

\subsection{Report Running Times}

Running time with all steps included:

\begin{center}
  \begin{tabular}{| l | r | r | r | r | r | r |}
    \hline
    Test Case \# & 01 & 02 & 03 & 04 & 05 & 06  \\ \hline
    1 Process/node &  &  &  &  &  &  \\ \hline
    2 Processes/node &  &  &  &  &  &  \\ \hline
    6 Processes/node &  &  &  &  &  &  \\ \hline
    12 Processes/node &  &  &  &  &  &  \\ \hline
  \end{tabular}
\end{center}


Running time excluding steps 1 \& 6:

\begin{center}
  \begin{tabular}{| l | r | r | r | r | r | r |}
    \hline
    Test Case \# & 01 & 02 & 03 & 04 & 05 & 06  \\ \hline
    1 Process/node &  &  &  &  &  &  \\ \hline
    2 Processes/node &  &  &  &  &  &  \\ \hline
    6 Processes/node &  &  &  &  &  &  \\ \hline
    12 Processes/node &  &  &  &  &  &  \\ \hline
  \end{tabular}
\end{center}

\end{document}
